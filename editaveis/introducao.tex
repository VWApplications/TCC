\chapter[Introdução]{Introdução}

Neste capítulo, é introduzido o trabalho de conclusão de curso. O capítulo está dividido em seções que estão dispostas da seguinte maneira: na seção 1.1 é contextualizado de forma resumida o que são as metodologias ativas de aprendizado até chegar no foco que é a metodologia \textit{Team Based Learning}; na seção 1.2 é apresentado a problemática; na seção 1.3 é apresentado a justificativa; na seção 1.4 os objetivos; na seção 1.5 é apresentado a metodologia de pesquisa e desenvolvimento e por fim, na seção 1.6 é apresentado a organização do trabalho.


\section{Contextualização}

A educação está em um impasse diante das várias mudanças acontecendo na sociedade atual. As instituições que ensinam e avaliam a todos de forma igual e exigem resultados previsíveis, ignoram o fato que hoje vivemos em uma sociedade do conhecimento. Essa sociedade é baseada em competências cognitivas, pessoais e sociais, que não se adquirem da forma convencional e que exigem proatividade, visão, personalização, colaboração e empreendedorismo \cite{moran}.

Na época em que o acesso à informação era difícil, fazia sentido os métodos tradicionais, que privilegiam a transmissão de informação pelos professores. Com a chegada da internet podemos aprender em qualquer lugar, a qualquer hora e com muitas pessoas diferentes. Essa mescla, entre sala de aula e ambientes virtuais é fundamental para abrir a escola para o mundo e trazer o mundo para dentro da escola \cite{moran}.

Dentro dessa era tecnológica em que vive a sociedade do conhecimento, o software faz parte do cotidiano das pessoas, podendo ser encontrado em hospitais, escolas,  atividades de lazer e de segurança, entre outros.  No entanto, a produção de software ainda enfrenta uma série de desafios, como redução de custos, cumprimento de prazos e alta obtenção de qualidade do produto final \cite{silva}.

Diante disso a Engenharia de Software apareceu para melhorar a sistematização do desenvolvimento de software, que deveria ser tratado como engenharia e não como arte. Desta  forma, a ideia foi propor a utilização de métodos, ferramentas e técnicas para a produção de software confiável, correto e entregue, respeitando os prazos e custos definidos \cite{soares}.

Não é de hoje que o mercado de engenharia exige dos seus profissionais algumas competências importante, como: capacidade de trabalho em equipe, análise de dados, resolução de problemas reais \cite{davis}. As metodologias ativas de aprendizagem têm como objetivo preencher essas lacunas que o mercado exige dos alunos para que esses se sintam melhor preparados para o mercado.

Apesar de pouco conhecidas, as metodologias ativas de aprendizagem já vêm sendo aplicadas de forma indireta pelos professores por meio de projetos, resolução de problemas e outros meios de ensinar e aprender que podem ser considerados como um tipo de metodologia ativa \cite{moran}.

Silberman, citado por \cite{barbosa} modificou, com suas palavras um provérbio chinês dito pelo filósofo Confúcio para facilitar o entendimento de métodos ativos de aprendizagem, que diz:

\begin{quote}
“O que eu ouço, eu esqueço; o que eu ouço e vejo, eu me lembro; o que eu ouço, vejo e pergunto ou discuto, eu começo a compreender; o que eu ouço, vejo, debato e faço, eu aprendo desenvolvendo conhecimento e habilidade; o que eu ensino para alguém, eu domino com maestria.”
\end{quote}

Essa citação, com a modificação de Silberman, resume a teoria por trás das metodologias ativas de aprendizagem. Se o ensino englobar as atividades de ouvir, ver,  perguntar, discutir, fazer e ensinar, estamos no caminho da aprendizagem ativa \cite{barbosa}.

Existem hoje diversas metodologias ativas de aprendizagem, por exemplo, o \textit{Team-Based Learning} ou TBL é uma metodologia de aprendizagem colaborativa, já o \textit{Problem-Based Learning}, conhecida como PBL, é mais focada na resolução de problemas complexos com um toque ainda na abordagem pedagógica de ensino tradicional, como slides, provas e etc \cite{cabrera}. O \textit{Project-Based Learning}, tem como base a resolução de desafios por meio de projetos. Além disso pode-se contar com o apoio de um método ativo bastante atual chamado Sala de Aula Invertida que nada mais é  do que substituir a maioria das aulas por conteúdos virtuais para que o tempo em sala seja otimizado \cite{moran}.

O principal foco deste trabalho é na metodologia ativa de aprendizado TBL, o qual  implementa o construtivismo e dá ênfase no papel do aluno como mestre de suas próprias experiências educacionais. Eles saem de agente passivo do conhecimento para um agente ativo \cite{gomez}.

\section{Problemática}

A maioria dos casos de uso do TBL utiliza-se de ferramentas de CMC (\textit{Computer Mediated Communication}) como fóruns de discussão, chats entre outros meios de comunicação eletrônicos como Facebook \cite{alhomod}, Blackboard, Moodle, Wiki ou Google Drive \cite{awatramani}, entre outras tecnologias web \cite{kam}. Às vezes, utilizam até mais de uma dessas ferramentas juntas.

Mais especificamente, as ferramentas de comunicação fornecem um meio que permite que grupos de alunos troquem idéias e opiniões e compartilhem informações a qualquer hora e em qualquer lugar. Os principais benefícios do uso de CMC são a conveniência, a independência do lugar, do tempo e o potencial para que os usuários se tornem parte de uma comunidade virtual \cite{berge}. Porém, essas ferramentas são uma extensão do uso do TBL e não uma forma de automatizar o uso do mesmo.

Um excelente exemplo de uso de ferramenta de CMC é o \textit{Moodle}, ela é uma boa ferramenta para disponibilizar materiais e notas, além de ter um pequeno fórum para discussão. Porém, muitas vezes esses tipos de ferramentas deixam a desejar em vários aspectos, por exemplo, ela não calcula as notas dos alunos e dos grupos automaticamente. Este cálculo é bem trabalhoso de se fazer já que o TBL é feito por pontuação e não por menção como é feito nas universidades e escolas, ele não gera relatórios para o professor saber qual é o tema que está gerando mais dúvidas entre os alunos, entre outros aspectos.

Com isso temos a seguinte questão: A automatização da metodologia ativa de aprendizado por uma ferramenta tornaria mais fácil a sua implantação em sala de aula? Tornando o processo mais prazeroso, tanto para o aluno quanto para o professor.

\section{Justificativa}

Procurando colaborar com essa problemática, o presente trabalho propôs a construção de uma plataforma para automatizar todo o processo da metodologia ativa de aprendizado: Team Based Learning, e com isso validar se a ferramenta estaria sendo efetiva no seu propósito.

\section{Objetivo}

O objetivo deste trabalho é a implementação da plataforma de gerenciamento da metodologia ativa de aprendizado: \textit{Team Based-Learning}, chamada PGTBL, para automatizar o uso dessa metodologia no ambiente acadêmico da FGA. Temos como objetivos específicos:

\begin{itemize}
  \item Estudar o contexto do Team Based Learning;
  \item Especificar os requisitos;
  \item Adaptar a metodologia ágil para as necessidades do projeto;
  \item Desenvolver a ferramenta e testá-la em um contexto real.
  \item Coletar feedback dos alunos comparando a utilização da metodologia antes e depois da ferramenta.
\end{itemize}

\section{Metodologia}

A metodologia utilizada para a implementação da ferramenta é a metodologia ágil com adaptações para a utilização do SCRUM, XP e SAFe para ser utilizado por um único desenvolvedor. Além disso, a ferramenta terá acesso livre para contribuições externas durante seu desenvolvimento, aplicando todas as boas práticas definidas na comunidade \textit{Open Source}.

Como base de sustentação do trabalho, foi realizada uma pesquisa bibliográfica com o intuito de reunir as informações e dados que servirão de base para a construção da ferramenta, utilizando como banco de dados a Scopus e o Google Acadêmico, além de documentações ou sites oficiais de ferramentas ou metodologias.

\section{Organização do Trabalho}

O trabalho está organizado da seguinte forma: No Capítulo 2 é apresentado o referencial teórico quanto ao contexto do TBL, gerenciamento utilizando scrum, xp e safe, qualidade de software utilizando como base o \textit{Goal Question Metric} (GQM) para coleta de métricas e alguns conceitos de testes, devops e software livre. No Capítulo 3 é apresentado a metodologia de desenvolvimento, ou seja, será apresentado passo a passo o processo na qual o desenvolvimento será realizado e a metodologia do trabalho de conclusão de curso (TCC). No Capítulo 4 é apresentado a proposta de trabalho que terá uma visão geral do produto a ser desenvolvido, além de algumas informações técnicas e de gerenciamento como \textit{Roadmap}, ferramentas, Estrutura análitica do projeto (EAP), arquitetura e requisitos iniciais. E por último, temos as Considerações Finais na qual terá um resumo geral do que foi falado, e o que se pode esperar do projeto no futuro.
