\begin{resumo}
O mercado de engenharia exige algumas competências que devem ser enfatizadas, entre elas destacam-se: capacidade de trabalho em equipe, análise de dados, resolução de problemas reais. As metodologias ativas de aprendizagem tem como objetivo preencher essas lacunas que o mercado exige dos alunos, entre elas temos o \textit{Team-Based Learning} (TBL) que é uma metodologia de aprendizagem colaborativa. Embora existam ferramentas para apoiar o uso do TBL com CMCs (\textit{Computer-Mediated Communication}), a literatura não trata de ferramentas para automatizar a execução e implantação do TBL no ambiênte acadêmico. O objetivo deste trabalho é a implementação de uma ferramenta chamada PGTBL para automatizar o uso dessa metodologia ativa de aprendizado. A metodologia de desenvolvimento utilizada para alcançar os objetivos foi uma adaptação ao SCRUM, XP e SAFe para uma equipe de apenas um membro que fará todos os papéis designados na metodologia. A finalidade do software é que a aplicação do TBL se torne algo mais fácil, constante e automatizado, tornando o processo mais prazeroso, tanto para o aluno quanto para o professor. A ideia do sistema é ter um design atrativo e será responsável por todo o processo da metodologia de ensino TBL.

 \vspace{\onelineskip}

 \noindent
 \textbf{Palavras-chaves}: TBL. metodologia ativa de aprendizado. aprendizado baseado em equipes. ferramenta. software
  livre. metodologia ágil. ensino. educação. automatização.
\end{resumo}
