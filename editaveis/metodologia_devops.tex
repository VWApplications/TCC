\subsection{Subprocesso de DevOps}

Relembrando que DevOps é a combinação de filosofias, práticas e ferramentas que aumentam a capacidade de distribuir aplicativos e serviços em alta velocidade. Essas velocidade permite que seus clientes sejam atendidos de forma melhor e as empresas conseguem competir de forma mais eficaz no mercado \cite{amazon}. No nosso processo de desenvolvimento chamamos esse subprocesso de operação e será realizado quase todo de forma automatizada de acordo com a figura \ref{fig:devops}

\begin{figure}[H]
	\centering
  \includegraphics[keepaspectratio=true,scale=0.4]{figuras/devops.eps}
  \caption[Subprocesso de DevOps.]{Subprocesso de DevOps. Fonte: Autor}
	\label{fig:devops}
\end{figure}

Esse subprocesso está ligado ao desenvolvimento da ferramenta e aos testes, ela começa com o desenvolvedor criando uma branch (ramificação) a partir da branch devel para criar a nova funcionalidade proposta de acordo com a política de branch estabelecida. Essa política estipula duas ramificações principais: a branch master que tem o código que irá para o ambiente de produção e a branch devel que irá para o ambiente de homologação. Esses ambientes são máquinas disponibilizadas remotamente para a instalação do software. A partir da devel será criado novas ramificações para novas funcionalidades.

Com a branch criada será desenvolvido o código da nova funcionalidade e assim que finalizado será enviado para repositório do github por meio da ferramenta de controle de versão de código git. No github o desenvolvedor irá abrir um Pull Request para que comece a roda o processo de integração contínua e deploy contínuo do software por meio da ferramenta Travis CI.

Na integração contínua será executado de forma automática os testes, além de verificar a qualidade do código por meio da ferramenta de análise estática Codacy. Com isso o desenvolvedor realizará o merge do código para a branch devel e desta para a branch master, em paralelo será realizado o deploy do software no ambiente de homologação e produção de acordo com a branch especificada.

O processo de deploy contínuo funciona da seguinte maneira: por meio de scripts o travis CI irá mandar a imagem de homologação ou produção para o repositório de imagens Dockerhub. Com isso o script irá entrar na máquina de homologação e atualizar o repositório com as novas modificações e disponibilizando a nova versão do software para uso em seu respectivo ambiente.
