\section{Software Livre}

De acordo com Richard Stallman (1996) citado por \cite{augusto} “Software livre não é software grátis” ou seja, o software livre não necessariamente precisa ser gratuito, ele traz à mente a liberdade e não o preço. O que caracteriza um software livre é a possibilidade de usar, modificar e redistribuir novas versões dos programas, na qual só é possível se o código fonte estiver disponível. Assim sendo, com posse do código-fonte, um desenvolvedor poderá estudar os programas, adaptá-los às suas necessidades, adicionar melhorias, criar e distribuir novas versões.

De acordo com \cite{augusto} somente nos últimos anos houve um aumento no interesse acadêmico e comercial sobre software livre. Diz ele que o software livre é tão antigo quanto a própria história da indústria do software.

O código-fonte é tornado público através de licenças específicas que garantem sua utilização, modificação e posterior redistribuição sem encargo algum, além de garantir e proteger o direito incondicional dos desenvolvedores. Um grande exemplo de software livre é o sistema operacional Linux \cite{augusto}.

\subsection{Motivadores}

De acordo com \cite{augusto} os principais motivadores que fazem os desenvolvedores brasileiros contribuírem ou criarem software livre são:

\begin{itemize}
  \item Aumentar o conhecimento em computação;
  \item Ganho de reputação por meio da contribuição de código ou da ajuda a outros usuários;
  \item Desenvolvimento em horas vagas sem pressões;
  \item Expectativas de que seja possível vender produtos ou serviços associados ao software livre.
\end{itemize}

\subsection{As quatro liberdades}

De acordo com o site do \cite{gnu} um programa é software livre se os usuários possuem as quatro liberdades essenciais:

\begin{enumerate}
  \item Liberdade de executar o programa como você desejar, ou seja, qualquer tipo de pessoa ou organização é livre para usá-lo em qualquer tipo de sistema computacional, ou para qualquer tipo de trabalho e propósito, sem que seja necessário comunicar ao desenvolvedor ou qualquer outra entidade específica. A liberdade de executar o programa como você desejar significa que você não está proibido ou impedido de executá-lo.
  \item Liberdade de estudar como o programa funciona, e adaptá-lo às suas necessidades, ou seja, acessar o código fonte é um pré-requisito.
  \item Liberdade de redistribuir cópias de modo a ajudar os outros.
  \item Liberdade de distribuir cópias de suas versões modificadas a outros, com isso você pode dar a comunidade a chance de beneficiar de suas mudanças. Ser livre para fazer tudo isso significa que você não precisa pedir ou pagar pela permissão para fazê-lo.
\end{enumerate}

\subsection{Fechamento da seção}

Essas liberdades podem ser restringidas de acordo com a licença imposta ao produto, por exemplo, se a licença do
programa diz que se você redistribuir o software ele terá que ter a mesma licença do software original, ou seja, você
não poderá pegar um software livre e tomar posse dele. Algumas licenças que são muito restritivas faz com que o software
não seja mais considerado livre \cite{gnu}.
